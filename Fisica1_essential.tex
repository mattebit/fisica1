\documentclass[a4paper]{report}

\usepackage[top=25mm,bottom=25mm]{geometry}
\usepackage[utf8]{inputenc}
\usepackage[italian]{babel}
\usepackage[T1]{fontenc}

\usepackage{amssymb}
\usepackage{mathtools}
\usepackage{graphicx}

\usepackage{hyperref}

\title{Dispense essenziali di Fisica 1}
\author{Matteo Bitussi \\ Laurea in Informatica, Unitn}
\date{Anno accademico 2019-2020}

\begin{document}
  \maketitle

  \tableofcontents

  \section*{Introduzione}
  Questa dispensa è pensata per raccogliere le informazioni essenziali necessarie per lo svolgimento degli esercizi durante l'anno e/o per l'esame finale. Per questo motivo non saranno approfondite e non potranno sostituire quelle fornite dal professore.

  \chapter{Cinematica del punto}
  \section{Moto rettilineo}
  \subsection{Velocità media}
  La velocità media $v_m$ del punto è il rapporto tra spostamento e tempo:
  \[ v_m = \frac{\Delta x}{\Delta t} = \frac{x_2 - x_1}{t_2 - t_1} \]
  Essa coincide con la definizione matematica di valor medio di una funzione in un dato intervallo
  \[ v_m = \frac{1}{t-t_0} \int_{t_0}^t v(t) dt \]
  \subsection{Velocità istantanea}
  La velocità istantanea  rappresenta la rapidità di variazione temporale della posizione nell'istante $t$ considerato. è data dalla derivata dello spazio rispetto al tempo
  \[ v = \frac{dx}{dt} \]
  \subsection{Leggi orarie}
  è la relazione generale che permette il calcolo dello spazio percorso nel moto rettilineo, qualunque sia il tipo di moto
  \[ x(t) = x_0 + \int_{t_0}^t v(t) dt \]
  $x_0$ rappresenta la posizione iniziale del punto, occupata nell'istante $t_0$
  \\
    Data l'accelerazione $a(t)$ si può ottenere la velocità $v(t)$ cioè vale la relazione
  \[ v(t) = v_0 + \int_{t_0}^t a(t) dt\]

  \subsection{Moto Rettilineo Uniforme}
  Nel MRU la velocità $v$ è costante
  \[ x(t) = x_0 + v(t-t_0)\]

  \subsection{Accellerazione media}
  Se tra gli istanti di tempo $t_1$ e $t_2$ la velocità varia da $v_1$ a $v_2$, si definisce \textbf{accellerazione media} del punto, il rapporto tra la variazione di velocità e l'intervallo di tempo
  \[ a_m = \frac{v_2-v_1}{t_2-t_1} = \frac{\Delta v}{\Delta t} \]

  \subsection{Accellerazione istantanea}
  \[ a = \frac{dv}{dt} = \frac{d^2x}{dt^2} \]

  \subsection{Moto rettilineo uniformemente accellerato}
  Se l'accellerazione di un moto è costante, questo si dice uniformemente accellerato, e la dipenmdenza della velocità dal tempo è lineare

  \[ v(t) = v_0 + a(t-t_0) \]
  \[ x(t) = x_0 + v_0(t-t_0) + \frac{1}{2} a(t-t_0)^2  \]

  \subsection{Moto armonico semplice}
  Un punto segue un moto armonico semplice quando la legge oraria è definita dalla relazione
  \[ x(t) = A sen(\omega t + \phi) \]
    Dove $A, \omega, \phi$ sono grandezze costanti: $A$ è detta \textbf{ampiezza del moto}, $\omega t + \phi$ \textbf{fase del moto}, $\phi$ \textbf{fase iniziale}, $\omega$ \textbf{pulsazione}.

    Il MAS è quindi un moto vario, dove tutte le grandezze cinematiche che lo descrivono $(x(t), v(t), a(t))$ variano nel tempo.
    Il periodo $T$ è
    \[ T = \frac{2\pi}{\omega} \]

    Si definisce \textbf{frequenza} $\nu$ del moto, il numero di oscillazioni in un secondo
    \[ \nu = \frac{1}{T}=\frac{\omega}{2\pi} \]
  \subsubsection{Velocità nel MAS}
  La velocità del punto che si muove con moto armonico si ottiene derivando $x(t)$:
  \[ v(t)=\frac{dx}{dt} = \omega A cos(\omega t + \phi) \]
  \subsubsection{Accellerazione nel MAS}
  Con un ulteriore derivazione si ottiene l'accelerazione del punto:
  \[ a(t)=\frac{dv}{dt}=\frac{d^2 x}{dt^2} = -\omega^2 A sin(\omega t + \phi) = -\omega^2 x(t). \]
  \subsubsection{Eq. Differenziale del moto armonico}
  La condizione necessaria e sufficiente perchè un moto sia armonico è data dall'equazione
  \[ \frac{d^2 x(t)}{dx^2} + \omega^2 x(t) = 0 \]

  \section{Moto Rettilineo Smorzato Esponenzialmente}
  Velocità del punto
  \[ v(t) = v_0 e^{-kt} \]
  Legge oraria
  \[ x(t) = \frac{v_0}{k} (1-e^{-kt}) \]

  \chapter{Dinamica del punto}
  \section{Leggi di Newton}
  \subsection{Prima legge di Newton o Legge di inerzia}
  Il legame tra la forza e lo stato del moto è data dalla \textbf{legge di Newton}
  \[ \vec{F} = m\vec{a} \]

  \subsection{Seconda legge di Newton}
  Esprime la legge fondamentale della dinamica del punto
  \[ \vec{F} = m\vec{a} = m\frac{d\vec{v}}{dt} = \frac{d^2\vec{x}}{dt^2} \]

  \subsection{Terza legge di Newton}
  Anche chiamato principio di azione e reazione delle forze
  \[ \vec{F}_{A \rightarrow B} = -\vec{F}_{B \rightarrow A} \]

  \subsection{Quantità di moto}
  Si definisce quantità di moto di un punto materiale il vettore
  \[ \vec{p} = m\vec{v} \]

  \subsection{Risultante delle forze}
  \[ \vec{R} = \vec{F}_1 + \vec{F}_2 + ... +\vec{F}_n = \sum_i \vec{F}_i \]

  \subsubsection{Equilibrio statico}
  Se un corpo è in \textbf{equilibrio statico} la sua risultante $\vec{R} = 0$

  \subsection{Reazioni vincolari}
  Sono reazioni dell'ambiente circostante

  \subsection{Forza Peso}
  La forza peso è proporzionale alla massa
  \[ \vec{P} = m\vec{g} \]




































  \subsection{Quantità di moto}
  Si definisce quantità di moto di un punto materiale il vettore:
  \[ \vec{p} = m \vec{v} \]
  \[ [\vec{p}] = \frac{m}{s} kg \]

  \chapter{Sistemi di riferimento}
  \subsection{Sistema di riferimento Inerziale}
  è un sistema di riferimento dove vale la prima legge della dinamica
  \subsection{Sistema di riferimento non Inerziale}
  La prima legge di Newton assume la forma
  \[ \vec{F} = m(\vec{a^1} + \vec{a_t} + \vec{a_c}) \]
  è presente una accellerazione di un corpo anche senza forze esercitate su di esso
  \subsection{Forze vere e forze apparenti}
  Se ci sono forze apparenti, allora il sistema di riferimento non è Inerziale

  \chapter{Dinamica del punto}

  \section{Risultante delle forze}
  La risultante delle forze è definita come la somma di tutte le forze applicate su un dato punto
  \[ \vec{R} = F_1 + F_2 + ... + F_n = \sum_{i} \vec{F_i} \]

  \subsection{Equilibrio statico}
  Se $R = 0$ (e il punto ha inizialmente velocità nulla) esso rimane in stato di quiete: sono realizzate le condizioni di \textbf{equilibrio statico} del punto.\\
  Devono quindi essere nulle le componenti della risultante, ovvero:
  \[ R = 0 \Rightarrow R_x = R_y = R_z = 0 \]

  \subsection{Reazione vincolare}
  Data la definizione di \textbf{equilibrio statico}, se un corpo soggetto all'azione di una forza, o della risultante non nulla di un insieme di forze, rimane fermo, dobbiamo dedurre che l'azione della forza provoca una reazione dell'ambiente circostante, detta \textbf{reazione vincolare}, che si esprime tramite una \textbf{eguale e contraria} alla forza, o alla risultante delle forze agenti.

  \section{Forza Peso}
  \[ P = mg \]

  \section{Impulso}
  \subsection{Impulso della forza}
  Si definisce impulso $ \vec{J} $ l'integrale della forza nel tempo:
  \[ \vec{J} = \int_0^t \vec{F} dt \]
  \[ [J] = N  s \]

  \subsection{Teorema dell'impulso}
  \[ \vec{J} = \int_0^t \vec{F} dt \int_{\vec{p}_0}^{\vec{p}} d\vec{p} = \vec{p} - \vec{p_0} = \Delta \vec{p} \]
  Se la massa $m$ è costante:
  \[ \vec{J} = \Delta \vec{p} = m \Delta v \]
  Se la forza $F$ è costante:
  \[ \vec{J} = \vec{F} \cdot t = \Delta p  \]

  \section{Lavoro}
  Il lavoro è pari all'integrale da $a$ a $b$ delle forze totali agenti sul corpo scalare lo spostamento $ds$
  \[ W = \int_{a}^{b} \vec{F_{tot}} \cdot \vec{ds} \]

  \section{Attrito}
  \subsection{Forza di attrito radente}
  E uguale a:
  \[ F_a = \mu_s \cdot N\]
  Dove $\mu_s$ è il coefficiente di attrito statico, e $N$ è la normale del corpo sul piano. $N$ si può anche esprimere come la componente ortogonale al piano della risultante delle forze che agiscono sul punto materiale che stiamo analizzando.
  \\Si ha una situazione di quiete quando la forza applicata $F$ è:
  \[ F \leq \mu_s N \]
  e una condizione di moto quando
  \[ F > \mu_s N \]

  \chapter{Termodinamica}
  \section{Trasformazioni cicliche}
  \subsection{Ciclo di Carnot}
  è un ciclo reversibile, la macchina è costituita da un gas \textbf{ideale}, trasformazione può essere espansione isoterma reversibile, espansione adiabatica reversibile, compressione isoterma reversibile, compressione adiabatica reversibile. Può essere rappresentato con solo due sorgenti. Si ha che per il rendimento in un ciclo di Carnot si ha
  \[ \eta = 1 - \frac{T_1}{T_2}\]
  Il risultato è vero anche per sistemi diversi dai gas ideali \\
  Vale inoltre
  \[ T_1 < T_2 \Rightarrow \eta < 1 \]
  e anche $Q_c$ calore assorbito, $Q_a$ calore ceduto,
  \[ |Q_c| = Q_1 < Q_2 = Q_a \]

  \section{Rendimento}
  \subsection{Macchine Termiche}
  \subparagraph{Macchine Frigorifere}







\end{document}
