\documentclass[a4paper]{report}

\usepackage[top=25mm,bottom=25mm]{geometry}
\usepackage[utf8]{inputenc}
\usepackage[italian]{babel}
\usepackage[T1]{fontenc}

\usepackage{amssymb}
\usepackage{mathtools}
\usepackage{graphicx}

\usepackage{hyperref}

\title{Dispense essenziali di Fisica 1}
\author{Matteo Bitussi \\ Laurea in Informatica, Unitn}
\date{Anno accademico 2019-2020}

\begin{document}
  \maketitle

  \tableofcontents

  \section*{Introduzione}
  Questa dispensa è pensata per raccogliere le informazioni essenziali necessarie per lo svolgimento degli esercizi durante l'anno e/o per l'esame finale. Per questo motivo non saranno approfondite e non potranno sostituire quelle fornite dal professore.

  \section*{General}
  \subsection{Prodotto tra vettori}
  $$ \vec{a} \times \vec{b} = \vec{c} $$
  $$ |\vec{c}| = |\vec{a}||vec{b}| sin \theta_{ab} $$
  Il vettore risultante ha direzione e verso dati dalla regola della mano destra, col pollice sul primo vettore a e indice sul secondo vettore b.

  \subsection{Proiezione di vettori}
  Supponiamo di avere un piano inclinato di $\alpha$ gradi, e un oggetto soggetto alla Forza Peso sul piano, essa seguirà la direzione dell'accelerazione di gravità, quindi, per scomporre la forza sul piano e sull'ortogonale del piano usiamo il fatto che gli angoli interni al rettangolo che si viene a creare dal vettore della forza sulle proiezioni di questo vettore sul piano e sull'ortogonale saranno $\alpha$.

  \chapter{Cinematica del punto}
  Parte della fisica che ha a che fare con la descrizione matematica del movimento dei corpi.

  \section{Concetto di movimento e moto}
  Possiamo descriverlo come la successione di posizioni nel tempo di un punto materiale. La linea rappresentata da questi punti viene chiamata \textbf{traiettoria}. Noi vogliamo descrivere matematicamente questo fenomeno.
  Per definire la posizione dobbiamo introdurre un sistema di riferimento.
  Chiamiamo \textbf{spostamento} in 1-Dimensione la distanza $$\Delta_x = x_a -x_b$$ tra due coordinate spaziali.\\
  Ad ogni evento fisico, ad esempio la posizione di un punto materiale, possiamo assegnare un tempo. Il "collegamento" tra la matematica dello spazio e la matematica del tempo è connessa dalla fisica.\\
  La rappresentazione del \textbf{moto} si effettua mettendo in relazione lo spazio con il tempo.\\
  La \textbf{cinematica} avviene quando da $x$ passo a $x(t)$ ossia $x$ in funzione del tempo. Anche: $$(t,x) \rightarrow (t, x(t))$$ ossia la coppia tempo,spazio viene mappata a tempo e spazio in funzione del tempo. Quindi non abbiamo più un campionamento di posizioni, ma abbiamo una funzione del tempo che descrive le posizioni

  \section{Unità di misura e grandezze fisiche}
  La \textbf{misura} è il \textbf{rapporto} tra la grandezza che sto considerando e una grandezza di riferimento che chiamo \textbf{unità di misura}
  Il concetto di definizione di unità di misura differisce da quello dell'analisi dimensionale, in quanto le u.d.m. che posso scegliere sono molteplici, a differenza dell'univoca grandezza fisica.
  Per esempio possiamo dire che:
  $$[\Delta x] = [x] = [L]$$
  e possiamo dire che
  $$u.d.m(\Delta x) = u.d.m.(x) = 1m (cm, mm, km, ..)$$

  \section{Moto rettilineo}
  \subsection{Velocità media}
  La velocità media $v_m$ del punto è il rapporto tra spostamento e tempo:
  $$ v_m = \frac{\Delta x}{\Delta t} = \frac{x_2 - x_1}{t_2 - t_1} $$
  Essa coincide con la definizione matematica di valor medio di una funzione in un dato intervallo
  $$ v_m = \frac{1}{t-t_0} \int_{t_0}^t v(t) dt $$
  Notiamo che $$[\Delta x] = [x] = [L]$$ cioè la lunghezza $L$. e \\
  $$ [\Delta t] = [t] = [T]$$ segue quindi
  $$ [v] = ... = [\frac{L}{T}]$$
  Dove le parentesi quadre sono la misura delle dimensioni che in output fornisce la grandezza fisica fondamentale a cui può essere ricondotta la grandezza fisica che abbiamo dato in input.
  \subsection{Velocità istantanea}
  La velocità istantanea  rappresenta la rapidità di variazione temporale della posizione nell'istante $t$ considerato. è data dalla derivata dello spazio rispetto al tempo. Semplicemente vogliamo considerare un piccolo intervallo di misurazione per avere la velocità istantanea. è importante che la funzione della velocità sia derivabile e continua
  $$ v(t) = \frac{dx}{dt} $$
  \subsection{Leggi orarie}
  è la relazione generale che permette il calcolo dello spazio percorso nel moto rettilineo, qualunque sia il tipo di moto
  $$ x(t) = x_0 + \int_{t_0}^t v(t) dt $$
  $x_0$ rappresenta la posizione iniziale del punto, occupata nell'istante $t_0$
  \\
    Data l'accelerazione $a(t)$ si può ottenere la velocità $v(t)$ cioè vale la relazione
  $$ v(t) = v_0 + \int_{t_0}^t a(t) dt$$

  \subsection{Moto Rettilineo Uniforme}
  Il MRU è rettilineo, avviene lungo una retta. Nel MRU la velocità $v = \frac{dx}{dt}$ è costante
  $$ x(t) = x_0 + v(t-t_0)$$

  \subsection{Accellerazione media}
  Se tra gli istanti di tempo $t_1$ e $t_2$ la velocità varia da $v_1$ a $v_2$, si definisce \textbf{accellerazione media} del punto, il rapporto tra la variazione di velocità e l'intervallo di tempo. Quindi l'accellerazione è la variazione di velocità nelt tempo
  $$ a_m = \frac{v_2-v_1}{t_2-t_1} = \frac{\Delta v}{\Delta t} $$
  L'analisi dimensionale dell'accellerazione è
  $$[a] = [\frac{\Delta v}{\Delta t}] = [\frac{v}{t}] = [\frac{\frac{L}{T}}{T}] = [\frac{L}{T^2}]$$
  e $$u.d.m.(a) = 1m/s^2$$ cioè un metro su secondo quadro

  \subsection{Accellerazione istantanea}
  $$ a = \frac{dv}{dt} = \frac{d^2x}{dt^2} $$

  \subsection{Moto rettilineo uniformemente accellerato}
  Se l'accellerazione di un moto è costante, questo si dice uniformemente accellerato, e la dipenmdenza della velocità dal tempo è lineare.
  $$ a = const$$
  $$ v(t) = v_0 + a(t-t_0) $$
  $$ x(t) = x_0 + v_0(t-t_0) + \frac{1}{2} a(t-t_0)^2  $$
  $t_0$ sarebbe l'istante temporale iniziale del moto. Invece $x_0$ sarebbe l'origine dell'asse di riferimento (tipo un offset)

  \section{Moto armonico semplice}
  Il moto armonico si definisce come un oscillazione sinusoidale. Possiamo immaginare il modo di una molla lasciata "rimbalzare" senza ostacoli e attriti.
  Un punto segue un moto armonico semplice quando la legge oraria è definita dalla relazione
  $$ x(t) = x_0 + A sin(\omega t + \phi) $$
  Dove $A, \omega, \phi$ sono grandezze costanti: $A$ è detta \textbf{ampiezza del moto} cioè la ampiezza di massima oscillazione, cioè la distanza tra il punto di partenza e il punto di massima oscillazione, $\omega t + \phi$ \textbf{fase del moto}, $\phi$ \textbf{fase iniziale}, $\omega$ \textbf{pulsazione}.
  $x_0$ è il punto di inizio. Il moto descritto è un moto periodico.

  Il MAS è quindi un moto vario, dove tutte le grandezze cinematiche che lo descrivono $(x(t), v(t), a(t))$ variano nel tempo.
  Il periodo $T$ è
  $$ T = \frac{2\pi}{\omega} $$

  Si definisce \textbf{frequenza} $\nu$ del moto, il numero di oscillazioni in un secondo. La frequenza è l'inverso del periodo
  $$ \nu = \frac{1}{T}=\frac{\omega}{2\pi} $$
  \subsubsection{Velocità nel MAS}
  La velocità del punto che si muove con moto armonico si ottiene derivando $x(t)$:
  $$ v(t)=\frac{dx}{dt} = \omega A cos(\omega t + \phi) $$
  \subsubsection{Accellerazione nel MAS}
  Con un ulteriore derivazione si ottiene l'accelerazione del punto:
  $$ a(t)=\frac{dv}{dt}=\frac{d^2 x}{dt^2} = -\omega^2 A sin(\omega t + \phi) = -\omega^2 x(t). $$
  \subsubsection{Eq. Differenziale del moto armonico}
  La condizione necessaria e sufficiente perchè un moto sia armonico è data dall'equazione
  $$ \frac{d^2 x(t)}{dt^2} + \omega^2 x(t) = 0 $$

  \section{Moti piani}
  Un vettore è una "freccia" che parte dal punto di applicazione. Il verso del vettore è dato dalla punta della freccia e la direzione dal gambo.
  Esistiono \textbf{grandezze fisiche scalari} o \textbf{grandezze fisiche vettoriali}.
  Le grandezze fisiche vettoriali sono ad esempio la posizione,
  $$ (\vec{x})$$
  Cioè il vettore che indica la posizione di un punto nel piano.
  Per definire lo \textbf{spostamento} non abbiamo più la differenza tra due punti, ma la differenza tra due vettori.
  $$\Delta x \rightarrow \Delta \vec{x}$$
  Da notare che il tempo non è una grandezza vettorizzabile, in quanto è rappresentato da un asse unidimensionale.\\
  La velocità sarà definita come:
  $$\vec{v} = \frac{d\vec{x}}{dt}$$.
  Stessa cosa per l'Accelerazione
  $$\vec{a} = \frac{d\vec{v}}{dt}$$
  Nella definizione di problemi in spazi multidimensionali, la differenza di un vettore può essere non solo nell suo modulo (intensità) ma anche nella sua direzione. Ad esempio l'accelerazione puo derivare da uno spostamento del vettore velocità.
  Le operazioni tra vettori sono la somma (metodo del parallelogramma o del punto-coda), il prodotto per uno scalare (vettore per numero il risultato è un vettore), la differenza tra vettori


  \section{Moto Rettilineo Smorzato Esponenzialmente}
  Velocità del punto
  $$ v(t) = v_0 e^{-kt} $$
  Legge oraria
  $$ x(t) = \frac{v_0}{k} (1-e^{-kt}) $$


  \section{Moto Circolare}
  \subsection{Moto Circolare Uniforme}
  Si dice circolare uniforme perchè la traiettoria è circolare, e uniforme perchè il modulo della velocità $|d\vec{v}|$ è costante.
  La \textbf{velocità angolare istantanea}, che è il rapporto tra la differenza dell'angolo e la differenza di tempo (ossia quanto cambia l'angolo rispetto al tempo), è
  $$ \omega = \frac{d\theta}{dt} $$
  $$ \omega = \frac{v}{R} $$
  Nel MCU essa è costante. Le leggi orarie per il MCU sono:
  $$ x(t) =x_0 + vt$$
  $$ \theta(t) = \theta_0 + \omega t$$
  Il MCU è un moto accellerato con accelerazione costane, \textbf{ortogonale alla traiettoria}, l'accelerazione si chiama centripeta
  $$ a= a_N = \frac{v^2}{R} = \omega^2 R$$
  Si tratta in oltre di un \textbf{moto periodico}, con periodo
  $$ T=\frac{2\pi R}{v} = \frac{2\pi}{\omega} $$
  La velocità tangenziale ha Modulo
  $$ v(t) = R \cdot \omega $$
  Essendo un moto periodico possiamo definire la frequenza sulla base del periodo:
  $$ f = \frac{1}{T}$$
  Da notare che è possibile stimare l'accelerazione di un corpo nella sua traiettoria interpretandola in un piccolo intorno come un moto circolare uniforme.
  \subsubsection{Grandezze angolari}
  Il movimento lungo la circonferenza si può descrivere sia dalla traiettoria lungo la circonferenza, sia dall'angolo individuato dalle varie posizioni. Quindi invece di parlare di spostamento lungo la circonferenza possiamo parlare di spostamento angolare
  $$ s \rightarrow \alpha$$
  $$ \Delta s \rightarrow \Delta \alpha$$
  $$ \frac{ds}{dt} = v \rightarrow \omega = \frac{d\alpha}{dt}$$
  $$[w] = [\frac{d\alpha}{dt}] = [\frac{1}{T}]$$
  $$ u.d.m.(\omega) = \frac{1 rad}{1 sec}$$
  Il legame tra le grandezze angolari e quelle "normail" è
  $$\frac{C}{C_g} = R$$
  Ossia, il rapporto tra la circonferenza e l'anglo giro è uguale al raggio

  \subsection{Moto Circolare non Uniforme}
  La velocità angolare istantanea è
  $$ \alpha = \frac{d\omega}{dt} = \frac{a_T}{R} $$
  dove $a_T$ è l'accelerazione tangenziale\\
  Le leggi orarie diventano:
  $$ w(t) = w_0 + \int_{t_0}^t \alpha(t) dt $$
  $$ \theta(t) = \theta_0 + \int_{t_0}^t \omega(t) dt $$

  \subsection{Forza (accelerazione) centripeta}
  Una forza è centripeta, se è ortogonale alla traiettoria descritta dal corpo su cui è applicata, ovvero se è ortogonale (normale) al vettore velocità
  $$ \vec{a_n} = -\omega^2 R \cdot \hat{n} = -\frac{v_T^2}{R} \cdot \hat{n} $$
  di conseguenza, la Forza
  $$ \vec{F_c} = ma_C =  -m\frac{v_T^2}{R} \cdot \hat{n}$$

  \subsection{Forza Centrifuga}
  $$ F_{centrif} = - F_{centripeta} $$


  \section{Moto parabolico}
  L'esempio più semplice è quello del lancio di un oggetto tenendo conto dell'accelerazione di gravità. Abbiamo una velocità iniziale $\vec{v_0}$, l'angolo di alzo $\alpha$ che è individuato dalla linea dell'orizzonte fino al vettore velocità $\vec{v_0}$, la gittata è il punto in cui il corpo lanciato ritorna a toccare il suolo (orizzonte).\\
  Si scompone il moto in base agli assi $x$ e $y$. Abbiamo che sull'asse delle $x$ non agisce acceleraziome quindi è un moto rettilineo uniforme (l'ungo quell'asse), l'unica accelerazione che abbiamo è quella nell'asse $y$.
  Per l'asse $y$ abbiamo che il moto è uniformemente accellerato (o decelerato) perchè l'unica accelerazione che agisce sul corpo è quella di gravità.
  Bisogna quindi scomporre la velocità iniziale nelle sue componenti $x$ e $y$: Ossia
  $$v_{0x} = v_0 cos\alpha$$
  $$v_{0y} = v_0 sin\alpha$$


  \chapter{Sistemi di riferimento}
  \subsection{Sistema di riferimento Inerziale}
  è un sistema di riferimento dove vale la prima legge della dinamica, cioè dove un corpo permane nel suo stato di moto (quiete) a meno dell'interazione con forze esterne.\\
  Un esempio è quello delle stelle fisse, ad esempio l'origine nel sole e gli assi verso altre galassie ferme.
  \subsection{Sistema di riferimento non Inerziale}
  La prima legge di Newton assume la forma
  $$ \vec{F} = m(\vec{a^'} + \vec{a_t} + \vec{a_c}) $$
  è presente una accellerazione di un corpo anche senza forze esercitate su di esso
  \subsection{Principio di relatività galileiana}
  Dice che è sempre possibile ottenere uno stato di quiete di un moto cambiano il sistema di riferimento che utilizzo purchè il cambiamento di moto sia esprimibile come velocità.
  $$\vec{v_0^i} = \vec{v} \Rightarrow \vec{v^'} = \vec{0}$$

  \subsection{Forze vere e forze apparenti}
  Se ci sono forze apparenti, allora il sistema di riferimento non è Inerziale. \\
  Le \textbf{forze vere} sono riconducibili alle interazioni fondamentali.\\
  Le \textbf{forze apparenti} sono una conseguenza del sistema di riferimento (relativo) scelto
  \subsection{Regola della vite}
  Il vettore $\vec{\omega}$ è un vettore che rappresenta la rotazione. La direzione identificata da $\vec{\omega}$ è l'asse di rotazione, il suo modulo sarà $\omega$, il verso è quello di avanzamento della vite.\\
  Usando la \textbf{formula di Poisson} possiamo concludere
  $$\frac{d\vec{A}}{dt} = \vec{\omega} \time \vec{A}$$
  \subsection{Teorema delle velocità relative}
  Avendo due sistemi di rferimento differenti per descrivere il moto di un corpo, supponendo che uno dei due sistemi si muova anch'esso, possiamo ricavare la velocità del moto in base alla rappresentazione col primo sistema e la velocità di trascinamento$ \vec{v_{o'}} + \vec{\omega} \time \vec{r^'}$ che deriva dal fatto che la misura sul secondo sistema è influenzata dal fatto che esso si muove. La velocità di trascinamento si può anche definire come la velocità relativa tra i due sistemi di riferimento, non centra la velocità del moto.
  $$ \vec{v} = \vec{v'} + \vec{v_{o'}} + \vec{\omega} \time \vec{r^'} = \vec{v'} + \vec{v_t}$$
  \subsection{Teorema delle accelerazioni relative}
  $$ \vec{a} = \vec{a'} + \vec{a_t} + \vec{a_c}$$
  dove $\vec{a_t}$ è l'accelerazione di trascinamento tra i due sistemi di riferimento, $\vec{a_c}$ è l'accelerazione di coriolì.


  \chapter{Dinamica del punto}

  \section{Const}
  \begin{itemize}
    \item Accelerazione gravitazionale $g = 9,80665 \frac{m}{s^2}$
  \end{itemize}

  \section{Forze fondamentali}
  \begin{itemize}
    \item Forza gravitazionale, descrive l'interazione tra le masse. Se un oggetto ha massa, subirà questa forza.
    \item Forza elettromagnetica, deriva dall'unificazione da due forze (elettrica e magnetica).
    \item Forza forte (o nucleare), responsabile della stabilità dei nuclei. L'interazione tra protoni e neutroni.
    \item Forza debole, responsabile delle interazioni per cui i nuclei cambiano di natura, responsabile del decadimento dei nuclei.
  \end{itemize}

  \section{Punto materiale}
  è un'approssimazione di un corpo in un punto. Quindi è un punto in cui si trova tutta la massa del corpo che sto considerando.\\
  Un \textbf{sistema di punti materiali} è un sistema che contiene vari punti materiali. Mi pongo il problema di differenziare le\\
  \textbf{forze esterne} dalle \textbf{forze interne} al sistema.
  Concentrandomi sul punto materiale $i$-esimo, posso scomporre la sua risultante delle forze in
  $$\vec{R_i} = \vec{R_i^{E}} + \vec{R_i^{I}} =  m_i \vec{a_i}$$
  Risultante delle forze esterne + risultante di quelle interne.\\
  Definiamo la risultante delle forze esterne come
  $$\vec{R} = M \frac{d^2\vec{x_{CM}}}{dt^2}$$
  Dove $M$ è la massa totale del sistema, e $\vec{x_{CM}}$ il centro di massa del sistema

  \section{Sistema isolato}
  In un sistema isolato, la risultante delle forze esterne è 0
  $$ \vec{R^E} = \vec{0}$$
  Quindi $$M \vec{a_{CM}} = 0$$
  Quindi la velocità del centro di massa è costante.\\
  In un sistema isolato la \textbf{quantità di moto} si conserva
  $$\vec{P} = M \vec{a_{CM}} = const$$
  Questo si chiama \textbf{Principio della conservazione della quantità di moto}, cioè in un sistema isolato la quantità di moto TOTALE si conserva nel tempo
  $$ \frac{d\vec{P}}{dt} = \vec{0}$$


  \section{Legge di gravitazione universale}
  Due punti materiali si attraggono con una forza di intensità indirettamente proporzionale al prodotto delle masse dei singoli corpi e inversamente proporzionale al quadrato della loro distanza:
  $$ F = - G \frac{m_1 \cdot m_2}{r^2} \hat{r_{12}} $$
  Dove G è la costante di gravitazione universale, e $\hat{r_{12}}$ è il versore che indica la congiungente dei due corpi. è importante dire che la forza in questione è sempre attrattiva, entrambi i corpi hanno questa forza che tende ad attrarli.\\
  è possibile calcolare la massa della terra o l'accelerazione gravitazionale usando la formula sopra
  $$g = \gamma \frac{M_T}{R^2}$$

  \section{Massa inerziale e gravitazionale}
  Massa inerziale per variazione dello stato di moto, l'accelerazione di gravità è una costante, non dipende dal moto del corpo. La massa inerziale è la capacità del corpo ad opporsi a sollecitazioni esterne. La massa gravitazionale è una costante di proporzionalità tra l'accelerazione che è comune a tutti gli oggetti. Se un corpo è soggetto ad una forza gravitazionale, allora possiede una massa gravitazionale.
  $$ m_i \vec{a} = m_G \vec{g} $$
  Posso quindi assumere che le due accelerazioni coincidano
  $$ \vec{a} = \vec{g} $$
  E che la massa inerziale sia uguale a quella gravitazionale
  $$ m_i = m_g$$

  \section{Leggi di Newton}
  \subsection{Prima legge di Newton o Legge di inerzia}
  La prima legge di Newton (o legge di Inerzia) afferma che un corpo permane nel suo stato di quiete, o di moto rettilineo uniforme a meno che non intervenga una forza esterna a modificarne tale stato.\\
  Un sistema di riferimento si dice \textbf{inerziale} se in esso vale la legge di inerzia.
  L'inerzia di un corpo è la sua capacità ad opporsi a cambiamenti se affetto da forze esterne.

  \subsection{Seconda legge di Newton}
  Esprime la legge fondamentale della dinamica del punto. Definisce il concetto di forza.
  $$ \vec{F} = m\vec{a} $$
  Dove $m$ è una costante, chiamata \textbf{massa inerziale}.
  $$ [F] = [M \frac{L}{T^2}]$$
  $$ u.d.m.(m) = 1Kg$$
  $$ u.d.m.(F) = 1 Kg \frac{m}{s^2} = 1N$$
  \subsubsection{Formulazione con la quantità di moto}
  Possiamo riformulare la seconda legge della dinamica usando la quantità di moto, cioè la forza è proporzionale alla derivata della quantità di moto nel tempo.
  $$ \vec{F} = \frac{d\vec{p}}{dt}$$

  \subsection{Terza legge di Newton o Legge di Azione Reazione}
  Anche chiamato principio di azione e reazione delle forze
  $$ \vec{F}_{A \rightarrow B} = -\vec{F}_{B \rightarrow A} $$

  \subsection{Quantità di moto}
  Si definisce quantità di moto di un punto materiale il vettore
  $$ \vec{p} = m\vec{v} $$
  Che è il prodotto tra massa e velocità

  \subsection{Risultante delle forze}
  Potremmo avere una situazione dove più forze interagiscono sul nostro corpo, la forza risultante da tutte le forze è la somma di tutte le forze (la somma dei vettori).
  $$ \vec{R} = \vec{F}_1 + \vec{F}_2 + ... +\vec{F}_n = \sum_i \vec{F}_i $$

  \subsubsection{Equilibrio statico}
  Se un corpo è in \textbf{equilibrio statico} la sua risultante delle forze $\vec{R} = 0$

  \subsection{Reazioni vincolari}
  Sono reazioni dell'ambiente circostante.\\
  Prendiamo il caso di una massa appoggiata su di un piano, ferma , allora esiste una forza vincolare
  $$ \vec{N} $$
  normale alla superficie di appoggio, che bilancia la forza peso agente sulla massa.

  \subsection{Forza Peso}
  La forza peso è proporzionale alla massa e all'acelerazione di gravità.
  $$ \vec{P} = m\vec{g} $$
  L'\textbf{Energia Potenziale} della forza peso per un punto $p$
  $$ U (\vec{}) = mg \vec{p} $$

  \section{Quantità di moto}
  Si definisce quantità di moto di un punto materiale il vettore:
  $$ \vec{p} = m \vec{v} $$
  $$ [\vec{p}] = \frac{m}{s} kg $$

  \section{Risultante delle forze}
  La risultante delle forze è definita come la somma di tutte le forze applicate su un dato punto
  $$ \vec{R} = F_1 + F_2 + ... + F_n = \sum_{i} \vec{F_i} $$

  \section{Equilibrio statico}
  Se $R = 0$ (e il punto ha inizialmente velocità nulla) esso rimane in stato di quiete: sono realizzate le condizioni di \textbf{equilibrio statico} del punto.\\
  Devono quindi essere nulle le componenti della risultante, ovvero:
  $$ R = 0 \Rightarrow R_x = R_y = R_z = 0 $$

  \section{Reazione vincolare}
  Data la definizione di \textbf{equilibrio statico}, se un corpo soggetto all'azione di una forza, o della risultante non nulla di un insieme di forze, rimane fermo, dobbiamo dedurre che l'azione della forza provoca una reazione dell'ambiente circostante, detta \textbf{reazione vincolare}, che si esprime tramite una \textbf{eguale e contraria} alla forza, o alla risultante delle forze agenti.

  \section{Forza Peso}
  $$ P = mg $$

  \section{Impulso}
  \subsection{Impulso della forza}
  Si definisce impulso $ \vec{J} $ l'integrale della forza nel tempo, cioè l'impulso è la variazione della quantità di moto nel tempo.
  $$ \vec{J} = \frac{|\vec{p_f} - \vec{p_i}}{\Delta t}$$
  $$ \vec{J} = \int_0^t \vec{F} dt $$
  $$ [J] = N  s $$
  Può anche essere scritto come differenza di quantità di moto:
  $$\vec{P} = \vec{p_1} - \vec{p_0} = \int_{t_0}^t \vec{F} dt $$

  \subsection{Teorema dell'impulso}
  $$ \vec{J} = \int_0^t \vec{F} dt \int_{\vec{p}_0}^{\vec{p}} d\vec{p} = \vec{p} - \vec{p_0} = \Delta \vec{p} $$
  Se la massa $m$ è costante:
  $$ \vec{J} = \Delta \vec{p} = m \Delta v $$
  Se la forza $F$ è costante:
  $$ \vec{J} = \vec{F} \cdot t = \Delta p  $$

  \section{Lavoro}
  Il lavoro è pari all'integrale da $a$ a $b$ delle forze totali agenti sul corpo scalare lo spostamento $ds$
  $$ W = \int_{a}^{b} \vec{F_{tot}} \cdot \vec{ds} $$
  Anche definito come:
  $$ W_{TOT} = E_{Kf} - E_{Ki} $$
  cioè il lavoro dal punto $i$ iniziale, al punto $f$ finale è definito come la differenza dell'energia cinetica nel punto $f$ meno quella nel punto $i$. Vale solo se il lavoro è \textbf{totale}, cioè se comprende tutte le forze agenti nel sistema\\
  Il lavoro si dice \textbf{motore} se la forza per cui lo calcolo partecipa all'accelerazione nel verso della velocità\\
  Il lavoro si dice \textbf{resistente} se la forza per cui lo calcolo partecipa alla decelerazione nel verso opposto della velocità
  abbiamo\\
  Il calcolo del lavoro è basato sul prodotto scalare $$\vec{a} \cdot \vec{b} = |\vec{a}| \ctdot |\vec{b}| \time cos \theta_{ab}$$

  $$ [W] = [F \cdot L] = [M \frac{L}{T^2} \cdot L] = [M \cdot \frac{L^2}{T^2}] = [E] $$
  $$ udm(W) = 1 Kg \frac{m^2}{s^2} = 1J$$
  \subsection{Lavoro Totale}
  $$ W_{TOT} = \sum_K W_K = \sum_K \int_i^f \vec{F}\cdot d\vec{s} $$
  \subsection{Energia meccanica}
  $$ W = \Delta E_K + \Delta E_P = \Delta E_m $$
  Nel caso di forze conservative, l'energia meccanica è Costante, quindi
  $$ E = E_K + U $$
  $$ E_{k_i} + U_i = E_{k_f} + U_f$$

  \section{Potenza}
  Mette in relazione il lavoro e il tempo durante il quale questo viene erogato
  $$ P = \frac{dW}{dt} = \vec{F} \cdot \vec{v} $$
  è il lavoro per l'unità di tempo, anche la potenza istantanea
  $$ u.d.m.[p] = 1 \frac{J}{s} = 1W $$
  Dove $1W$ significa 1 Watt\\
  La potenza media, invece:
  $$ \overline{p} = \frac{W}{\Delta t} =  $$


  \section{Forze conservative}
  Esstono delle forze, per cui il lavoro per andare dal punto iniziale a quello finale \textbf{non dipende dal percorso fatto}. Queste forze si dicono conservative\\
  Quindi, indipendentemente dal percorso potrò calcolare il lavoro Come
  $$ W_{i \rightarrow f} = \int_{p_i}^{p_f} \vec{F} \cdot d\vec{s} = U(p_f) - U(P_i)$$
  Una forza $\vec{F}$ è conservativa, se il lavoro compiuto lungo un percorso chiuso è nullo (cioè se torno da dove sono partito), indipendentemente dal percorso scelto
  $$ W_{chius} = 0 \quad \oint \vec{F} \cdot \vec{ds} = 0 $$
  Una forza si dice conservativa se si può scrivere come derivata di un altra funzione di energia potenziale (?)
  $$ F= -\frac{dU}{dx} (\vec{F} = \frac{-\vec{d} U}{dx})$$
  \paragraph{Corollario}
  $$ W_{a \rightarrow b} = \int_a^b \vec{F}\cdot \vec{ds} $$
  non dipende dal percorso scelto

  \section{Legge della conservazione della quantità di moto}
  Stabilisce che la quantità di moto di un sistema isolato è costante nel tempo (si conserva)
  $$ p = m_1v_1 + m2_v2 ... + m_iv_i $$
  oppure
  $$ \frac{dp}{dt} = 0 $$

  \section{Forza elastica}
  é la forza esercitata da una molla che si oppone ad una forza esterna. La forza elastica è proporzionale a quanto comprimo o allungo la molla $\Delta \vec{l}$
  Quindi la forza elastica è
  $$ \vec{F_{el}} = -K(\vec{l} - \vec{l_0}) = -K \Delta \vec{l} $$
  $K$ è la costante elastica.
  $$[K] = [\frac{F}{L}], \quad u.d.m.(K) = 1 \frac{N}{m}$$
  Possiamo supporre un corpo attaccato ad una molla appoggiato su di iun piano, ignorando gli attriti fissiamo il sistema di riferimento sulla posizione del corpo con la molla a riposo. Quindi possiamo scrivere l'equazione del moto come:
  $$x(t) = A sin(\omega t + \phi)$$
  Dove $w^2$ nel nostro caso è uguale a
  $$ w^2 = \frac{K}{m} $$
  che derivando risulta
  $$ v(t) = + A \omega cos(\omega t + \phi) $$
  $$ a(t) = -A \omega^2 sin (\omega t + \phi) $$
  Abbiamo anche che il \textbf{lavoro della forza elastica} sarà
  $$ \int_i^f \vec{F_el} \cdot d\vec{r} = - \frac{1}{2} K (x_f^2 -x_i^2)$$
  L'\textbf{Energia potenziale} della forza elastica in un certo punto $P$ è
  $$ U_{el}(\vec{P}) = \frac{1}{2} K x^2$$


  \section{Attrito}
  \subsection{Forza di attrito radente}
  E uguale a:
  $$ F_a = \mu_s \cdot N$$
  Dove $\mu_s$ è il \textbf{coefficiente di attrito statico}, e $N$ è la normale del corpo sul piano. $N$ si può anche esprimere come la componente ortogonale al piano della risultante delle forze che agiscono sul punto materiale che stiamo analizzando.
  \\Si ha una situazione di quiete quando la forza applicata $F$ è:
  $$ F \leq \mu_s N $$
  e una condizione di moto quando
  $$ F > \mu_s N $$
  Il \textbf{lavoro della forza di attrito} è sempre negativo,
  $$ W = \int \vec{F_as} \cdot d\vec{r}$$
  \paragraph{Coefficiente di attrito dinamico}
  Una volta "battuto" l'attrito statico, il moto continua ad essere rallentato dall'attrito radente, ma con un coefficiente di attrito diverso chiamato dinamico
  $$ \mu_d $$


  \section{Pendolo semplice}
  Supponiamo un pendolo che continui in moto perpetuo perchè privo di attriti. Le forze che agiscono sulla massa m attaccata al filo sono la massa stessa del peso e la tensione del filo. Il sistema di riferimento è basato sulla tangente alla traiettoria in concomitanza del punto e perpendicolare a questa tangente. Avremmo quindi la forza peso decomposta in base agli assi.
  Abbiamo anche che lungo la tangente per definire la velocità con cui il corpo si muove posso usare la formula del moto circolare, avendo lunghezza (raggio) del pendolo costante
  $$ v = \omega R$$
  $$- |F_pparall| = - mg sin \theta = m a_T = m \frac{dv}{dt} = m \frac{d(l\omega)}{dt} = ml \frac{d^2 \Theta}{dt^2}$$
  Semplificando le masse e spostando le cose
  $$ \frac{d^2\Theta}{dt^2} + \frac{g}{l} sin \Theta = 0 $$
  Come per la molla abbiamo che possiamo scrivere
  $$ \frac{g}{l} = \omega^2$$

  \section{Energia}
  \subsection{Energia Cinetica}
  è definita come
  $$ E_K = \frac{1}{2}m v^2 $$
  $$ dW = d[\frac{1}{2}m v^2] $$
  l'energia cinetica è sempre positiva, dipende solo dalla velocità dell'oggetto (e quindi dal sistema di riferimento che stiamo usando)
  $$ u.d.m.[E_K]= 1J = 1N\cdot M = 1 Kg \frac{m^2}{s^2}$$
  Se tengo in considerazione tutte le forze che determinano il mio movimento, e ho che vale la seconda legge della dinamica, allora il lavoro della risultante di queste forze è uguale alla differenza tra lo stato finale e quello iniziale di una certa quantità, che chiamo energia cinetica.

  \subsection{Energia potenziale}
   l'energia potenziale di un oggetto è l'energia che esso possiede a causa della sua posizione.
   Se la forza è conservativa, si può sempre definire una funzione della posizione che si chiama energia potenziale.
   Possiamo, quindi, scrivere il lavoro come
  $$ W_{AB} = -(U_B - U_A) $$
  \paragraph{Definizione}
  $$ U(\vec{x_P}) = - \int_o^P \vec{F}\cdot \vec{ds} $$
  \paragraph{Lavoro e energia potenziale}
  $$ W_{a\rightarrow b} = -\Delta U $$

  \subsubsection{Energia potenziale (spiegazione2)}
  Nel caso il campo di forze sia conservativo, il lavoro non dipende dal percorso scelto, ma solo dagli estremi del cammino. Il differenziale $dW$ è un diff. perfetto, quindi si ha:
  $$W = \oint_a^b F \cdot dx = -[U(a) - U(b)] = -\Delta U$$
  Nel caso più semplice, in cui il moto si svolge in una sola direzione:
  $$ F(x) = -\frac{d}{dx}U(x) $$
  \begin{itemize}
    \item E.p. Elastica = \( U(x) = \frac{1}{2} K (x-x_0 )^2 \)
  \end{itemize}

  \subsection{Energia meccanica}
  è la somma di energia cinetica ed energia potenziale attinenti dallo stesso sistema.\\
  Per \textbf{forze conservative} è sempre la stessa durante l'evoluzione del moto
  $$ E_m = E_K + E_P $$
  Si definisce \textbf{Conservazione dell'energia meccanica}
  $$ \Delta U = \Delta E_k$$
  Cioè la differenza dell'energia potenziale finale meno iniziale  deve essere uguale alla differenza dell'engergia cinetica finale meno l'iniziale
  \paragraph{Lavoro di forze non conservative}
  Per forze non conservative
  $$ W_{n.c.} = \Delta E_K + \Delta E_P = \Delta E_m $$

  \section{Forza apparente}
  In meccanica classica, un'interazione apparente, detta anche interazione fittizia o inerziale, è una forza, o un momento, che, anche se non vi viene applicata direttamente, agisce su un corpo al pari delle forze e dei momenti reali, o effettivi

  \section{Urti}
  Un urto è l'interazione d inamica tra due masse, vengono in contatto tra di loro e modificano il loro stato di moto.
  L'urto trasferisce quantità di moto da una massa all'altra. Quello che succede nell'urto viene semplificato, supponendo che il tempo in cui accade è molto minore del tempo del moto delle masse.
  L'urto generico è descritto come
  $$ \vec{P_i} = \vec{P_f}$$
  $$ \vec{P_{1i}} + \vec{P_{2i}} = \vec{P_{1f}} + \vec{P_{2f}}$$
  QUindi
  $$m_1 \vec{v_{1i}} + m_2 \vec{v_{2i}} = m_1 \vec{v_{1f}} + m2 \vec{v_{2f}}$$
  \subsection{Urto perfettamente anelastico}
  Due oggetti (masse)$m_1, m_2$ si scontrano, e ne risulta un unica particella $m_3$\\
  Dalla conservazione della quantità di moto abbiamo
  $$ \vec{P_i} = \vec{P_f} $$
  e quindi
  $$ \vec{P_{1,i}} + \vec{P_{2,i}} = \vec{P_{3,f}} $$

  \subsection{Urto elastico}
  Nell'urto elastico vale
  $$\vec{P_i} = \vec{P_f}$$
  Ma vale Anche
  $$E_{ki} = E_{kf}$$




  \section{Leggi di Keplero}
  \subsection{Prima}
  Dice che i moti dei pianeti avvengono su orbite ellittiche, in cui il sole occupa il fuoco dell'ellisse
  \subsection{Seconda}
  Dice che le orbite avvengono in modo che la velocità areolare sia Costante, la velocità areolare è l'area spazzata dall'orbita nel tempo dt, individuata con un "triangolo".
  $$ \frac{dA}{dt} = const$$
  \subsection{Terza}
  Il rapporto tra il quadrato del periodo dell'orbita del pianeta e il cubo del semiasse maggiore dell'ellisse è costante.
  $$ \frac{T^2}{a^3} = K $$





  \chapter{Termodinamica}
  Il \textbf{tempo} non riveste un ruolo importante come in meccanica. La durata delle trasformazione non è essenziale alla descrizione

  \section{Const}
  \begin{itemize}
    \item $T_0 = 273,16 K$ Punto triplo $H_2O$
    \item $t(^\circ C)= T(K) - 273,16$
    \item $t(^\circ F)= \frac{9}{5} t(^\circ C) + 32$
    \item $R = 8,314 \frac{J}{K \cdot mol}$
    \item $1 m^3 = 1000l$
    \item $1 atm = 101325Pa = 1,013 \cdot 10^{-5}$
    \item Gas monoatomico: $\frac{3}{2}R$
    \item Gas diatomico: $\frac{5}{2}R$
    \item Gas poliatomico $3 R$

    Tipi di grandezze:
    \begin{itemize}
      \item Grandezze \textbf{estensive}, sono grandezze che dipendono dall'estensione del sistema termodinamico (per esempio volume o massa)
      \item Grandezze \textbf{intensive} non di pendono dall'estnesione del sistema termodinamico (ad esempio pressione, temperatura e densità)
    \end{itemize}
  \end{itemize}

  \section{Tipi di scambi}
  Ci possono essere diversi tipi di scambi tra un sistema termodinamico e l'ambiente. Lo scambio può essere definito come
  \begin{itemize}
    \item Aperto, se c'è sia scambio di materia che di energia (per esempio una pentola d'acqua che bolle senza coperchio)
    \item Chiuso, se non c'è scambio di materia ma c'è di energia (per esempio pentola d'acqua che bolle con coperchio)
    \item Isolato, se non scambia ne materia ne energia (per esempio contenitore esolato che contiene acqua)
    \item Impossibile scambiare materia senza energia.
  \end{itemize}

  \section{Equilibrio Termodinamico}
  Quando sussiste un equilibrio \textbf{meccanico}, un equilibrio \textbf{chimico} (no reazione chimica) e un equilibrio \textbf{termico} (no differenza di temperatura) all'interno di un sistema termodinamico.\\
  Da notare che questo deve valere per ogni "parte" del mio sistema che prendo in considerazione.
  \subsection{Principio zero della termodinamica}
  se ho un sistema $T_A$ che è in equilibrio termico col sistema $T_B$..
  $$T_A = T_B$$
  e $$T_B = T_C$$
  allora $$T_A = T_C$$


  \section{temperatura}
  Definita dalla media dell'energia cinetica dei componenti interni. La temperatura definisce varie esperienze nel mondo: espansione/contrazione liquidi e gas, variazione resistenza elettrica, ...\\
  è definita su una scala avente come zero il punto triplo dell'acqua
  $$T_0 = 273,16 K = 0 °C$$
  Quindi si può dire che la temperatura in gradi centigradi
  $$ t(°C) = T(K) - 273,16$$
  E in gradi farenhait
  $$ t(°F) = \frac{9}{5} t(°C) + 32$$
  La temperatura è misurata col \textbf{termometro}, il quale trasforma i cambiamenti di temperatura in cambiamenti di spazio
  $$\Delta T \rightarrow \Delta x$$


  \section{Contatto termico}
  Avendo dei corpi a contatto tra di loro, se essi hanno temperatura inziale diversa tra di loro, per un tempo $\Delta t$ atteso troveremo i corpi con temperatura uguale, indipendentemente dal tipo di contatto.\\
  Tipi di Contatto
  \begin{itemize}
    \item Parete diatermica: Ottimo contatto tra i corpi, il calore si scambia facilmente
    \item Pareti adiabatiche (isolate): non c'è scambio di calore (in un tempo limite)
  \end{itemize}

  \section{Legge di Joule}
  Qualunque tipo di lavoro venga effettuato in un sistema, questo risulterà in un aumento di temperatura del sistema


  \section{Primo principio della Termodinamica}
  $$ \Delta U = Q - W $$
  Dove, $\Delta U$ è la variazione di energia interna al sistema, $Q$ è il calore ($>0$ assorbito/ricevuto, $<0$ ceduto), e $W$ è il lavoro, ($>0$ è compiuto, $<0$ è "ricevuto".\\
  Il lavoro $W$ e il calore $Q$ dipendono dal percorso, $\Delta U$ no.

  \section{Trasformazioni termodinamiche}
  una trasformazione è il passaggio di un sistema termodinamico da uno stato ad un altro. In genere durante una trasformazione il sistema termodinamico non è in equilibrio

  \subsection{Trasformazioni quasi-statiche}
  Sono passaggi da uno stado d'equilibrio ad un altro stato di equilibrio tramite uno stato di equilibrio. Posso immaginare che le cose siano fatte con una calma e dolcezza tale da non perturbare l'ambiente.\\
  Se una trasformazione è quasi-statica (e non ci sono dissipazioni) allora è una trasformazione reversibile.

  \section{Trasformazioni cicliche}
  \subsection{Ciclo di Carnot}
  è un ciclo reversibile, la macchina è costituita da un gas \textbf{ideale}, la trasformazione può essere
  \begin{itemize}
    \item espansione isoterma reversibile
    \item espansione adiabatica reversibile
    \item compressione isoterma reversibile
    \item compressione adiabatica reversibile
  \end{itemize}
  Può essere rappresentato con solo due sorgenti. Si ha che per il rendimento in un ciclo di Carnot si ha
  $$ \eta = 1 - \frac{T_1}{T_2}$$
  Il risultato è vero anche per sistemi diversi dai gas ideali \\
  Vale inoltre
  $$ T_1 < T_2 \Rightarrow \eta < 1 $$
  e anche $Q_c$ calore assorbito, $Q_a$ calore ceduto,
  $$ |Q_c| = Q_1 < Q_2 = Q_a $$

  \subsection{Cicli irreversibile}
  Ciclo è irreversibile quando almeno un tratto del ciclo è irreversibile

  \subsection{Trasformazione reversibile}
  Se è possibile riportare allo stato iniziale sia il sistema, che l'ambiente esterno.
  \begin{itemize}
    \item Trasformazione quasi-statica
    \item Non cisono dissipazioni
  \end{itemize}

  \section{Calore}
  Il calore è qualcosa che determina un cambiamento di temperatura.
  $$ \Delta T = \gamma Q$$
  Il calore è anche proporzionale all'energia interna del sistema
  $$ \Delta U = Q$$
  Abbiamo anche che il calore è equivalente al lavoro che ho applicato al Sistema
  $$Q = -W = \Delta U_{int}$$

  \subsection{Calore specifico}
  $$ Q = cm(T_f - T_i) $$
  Segue, c è il calore specifico del corpo
  $$ c = \frac{1}{m} \cdot \frac{dQ}{dT} $$
  Più precisamente, lungo il percorso $\gamma$
  $$ c_\gamma = \frac{1}{m} [\frac{dQ}{dT}]_\gamma $$
  $$ udm[c] = \frac{J}{kg\cdot K} $$
  è una grandezza \textbf{intensiva} (dipende dalla massa)\\
  Dove $c$ è una costante che dipende dal materiale, $m$ è la massa del materiale, e $T_f - T_i$ la differenza di temperatura
  \subsubsection{nota}
  Il calore specifico di un corpo è utile solo se $dT \neq 0$
  \subsubsection{nota 2}
  $$ Q = \int_{a}^b dQ = \int_a^b m c_\gamma[T] dT $$
  \subsection{Capacità termica}
  $$ Q = C(T_f-T_i) $$
  Dove $C = mc_\gamma$, quindi
  $$ C_\lambda = [\frac{dQ}{dT}]_\gamma $$
  è una grandezza \textbf{estensiva}.
  \subsection{Calore specifico a volume costante}
  \begin{itemize}
    \item Per i gas \textbf{monoatomici} (ideali):  $c_V = \frac{3}{2}R$ dove $3$ sono i gradi di libertà della molecola
    \item (alcuni) gas \textbf{biatomici} (ideali):  $c_V = \frac{5}{2}R$
  \end{itemize}
  \subsection{Calore specifico a pressione costante}
  $$ c_p = R + c_V $$

  \section{Cambi di fase}
  Gli stati di aggregazione della materia Sono
  \begin{itemize}
    \item Solido: ha volume e forma propri
    \item Liquido: ha volume proprio, ma non ha forma propria
    \item Gassoso: non ha volume proprio e neanche forma propria
  \end{itemize}
  I passaggi sono:
  \begin{itemize}
    \item solido $\rightarrow$ liquido = liquefazione
    \item liquido $\rightarrow$ solido = solidificazione
    \item liquido $\rightarrow$ gassoso = evaporazione
    \item gassoso $\rightarrow$ liquido = condensazione
    \item solido $\iff$ gassoso = sublimazione
  \end{itemize}
  \subsubsection{Punto triplo}
  Chiamiamo punto triplo per una data sostanza, il punto di intersezione delle tre linee di cambio di fase, dove le tre fasi possono coesistere.
  \subsubsection{Punto critico}
  è un punto che demarca la fine della linea di separazione delle fasi liquide e gassose, oltre questo punto non è più possibile distinguere la fase liquida da quella gassosa.
  \subsection{Calore latente}
  Nel cambio di fase fusione e evaporazione
  $$ Q = \lambda m$$
  dove $\lambda$ è il calore latente.\\
  Per l'acqua
  $$ \lambda_{fusione}^{H_2O} = 3,3 \cdot 10^5 \frac{J}{Kg} \quad @ 273,16 K $$
  $$ \lambda_{fusione}^{H_2O} = 22,6 \cdot 10^5 \frac{J}{Kg} \quad @ 373,16K $$

  \section{Trasmissione del calore}
  I modi di trasmissione del calore sono
  \begin{itemize}
    \item conduzione
    \item convezione
    \item irraggiamento
  \end{itemize}
  \subsection{Conduzione}
  $$ dQ = -K \frac{dT}{dZ} ds dt $$
  dove $K$ esprime la conducibilità del materiale
  $$ udm[K] = \frac{J}{m\cdot sK} $$
  \subsection{Convezione}
  Il calore si trasmette mediante lo spostamento delle parti calde del sistema (esempio: acqua)
  \subsection{Irraggiamento}
  Un sistema emette e assorbe onde elettromagnetiche, la legge che lega l'energia che viene emessa tramite onde elettromagnetiche e la temperatura del corpo è la legge di Stefan-Boltzmann
  \subsubsection{Legge di Stefan-Boltzmann}
  Dice che
  $$\epsilon = \sigma e T^4$$
  $$ udm[\epsilon] = [\frac{E}{L^2T}] $$
  Dove $\epsilon$ è il potere emissivo del corpo, $e$ l'emissività del corpo (la capacità di un corpo di emettere onde) e $\sigma$ è la costante di Stefan-Boltzmann
  $$ \sigma = 5,67 \cdot 10^{-8} \frac{J}{m^2sK^4} $$
  \subsubsection{nota}
  è improprio considerare l'irraggiamento uno scambio di calore
  \subsection{Vaso DEWAR}
  Ottimo recipiente isolante

  \section{Equazione di stato di Gas Ideali}
  $$ pV = nRT $$
  \subsection{Isocora}
  Definita da Gay-Lussac, il volume $V$ è costante,
  $$ p = p_0(1 + \beta t) $$
  \subsection{Isobara}
  Definita da Gay-Lussac, la pressione $p$ è costante,
  $$ V = V_0(1 + \alpha t) $$
  \subsection{Isoterma}
  Definita da Boyle, la temperatura $t$ è costante,
  $$ p_i V_i = p_f V_f = cost$$
  \subsection{Lavoro nei Gas}
  $$ W = \int_i^f p(V)dV $$
  \subsection{Energia interna nei Gas ideali}
  Sperimentalmente, per  un gas ideale,
  $$ U = U(T) $$
  \subsection{Equazione di stato dei gas non ideali}
  Van der Wals
  $$ f(p,V,T) = 0 $$

  \section{Trasformazioni dei gas ideali}
  Per gas ideali vale sempre
  $$ dU = n c_V dT $$
  \begin{center}
  \begin{tabular}{ |c|c|c|c|c| }
   \hline
    & Isocora & Isobara & Isoterma & Adiabatica \\
    & $\Delta V = 0$ & $\Delta p = 0$ & $\Delta T = 0$ & -\\
   \hline
   $\Delta U$ & $n c_V \Delta T$ & $n c_V \Delta T$ & $n c_V \Delta T = 0$ & $n c_V \Delta T$ \\
   $Q$ & $n c_V \Delta T$ & $n c_p \Delta T$ & $nRTln(\frac{v_f}{v_i})$ & $0$\\
   $W$ & $0$ & $ nR\Delta T \quad p_B(\Delta V)$ & $nRTln(\frac{v_f}{v_i})$ & $-n c_V \Delta T$ \\
   \hline
  \end{tabular}
  \end{center}
  \subsection{Trasformazione Isocora}
  Siccome il volume $V$ è costante, il lavoro $W = 0$, quindi
  $$ dU=dQ $$
  $$ dQ = nc_vdT = dU$$
  Dove $c_v$ è il calore specifico a volume costante
  \subsection{Trasformazione Isobara}
  $$ dU = dQ-dW $$
  $$ n c_V dT = n c_p dT - nR dT \Rightarrow \quad c_p -c_V = R $$
  che è la relazione di Mayer. dove c $c_p$ è il calore specifico a pressione costante
  \subsection{Trasformazione Isoterma}
  $$ \Delta U = 0 \quad \Rightarrow \quad Q-W =0 $$
  Quindi
  $$ Q = W = n R T ln(\frac{V_f}{V_i}) $$
  \subsection{Trasformazione Adiabatica}
  $$ p V^{\gamma} = cost $$
  con $\lambda$ che varia in base al tipo di gas
  \subsection{Trasformazione Generica}
  $$ dQ = nc_V dT + dW $$

  \section{Trasformazioni cicliche}
  Lo stato iniziale $A$ coincide con lo stato finale $B$
  $$ \Delta U = \int_A^B dU = \oint dU = U_B - U_A = 0 $$
  $$ Q = \int_A^B dQ = \oint_{A \rightarrow B} dQ = Q $$
  $$ W = \int_A^B dW = \oint dW = W $$
  $$ \Delta U = 0 \Rightarrow Q = W $$
  $$ Q = Q_C + Q_A,  \quad Q_C < 0, \quad Q_A > 0 $$
  Dove $Q_C$ è il calore ceduto, $Q_A$ quello assorbito.\\
  Stessa cosa per il lavoro,
  $$ W = W_F + W_S, \quad W_F > 0  \quad W_S < 0 $$
  Dove $W_F$ è il lavoro fatto, e $W_S$ quello subito
  \subsection{Macchine termiche}
  il lavoro $W$ compiuto è $>0$,
  $$ W>0 \quad \Rightarrow \quad Q > 0 $$
  \subsection{Macchien Frigorifere}
  $$ W<0 \quad \Rightarrow \quad Q < 0 $$

  \subsection{Ciclo di Otto}
  Tipico del motore a scoppio. è caratterizzato da
  \begin{itemize}
    \item espansione isobara
    \item compressione adiabatica
    \item accensione/combustione
    \item decompressione
    \item scarico
  \end{itemize}
  Il rendimento è pari a
  $$ \eta = 1 - \frac{|T_a - T_d|}{T_c - T_b} $$

  \subsection{Ciclo di Carnot Inverso}
  Verso di percorrenza inverso del Ciclo di Carnot, il lavoro è negativo \\
  Rendimento
  $$ \xi = \frac{T_1}{T_1 - T_2} $$

  \section{Rendimento}
  $$ \eta=\frac{W}{Q_A} \quad = 1- \frac{|Q_C|}{Q_A}$$
  \subsection{Macchine Termiche}
  Rendimento sempre compreso tra
  $$0 \leq \eta < 1$$
  e quindi
  $$0 \leq W < Q_A $$
  e quindi
  $$ 0 < |Q_C| < Q_A $$
  \subsection {Macchine Frigorifere}
  L'efficienza, o \textbf{Coefficiente di prestazione}
  $$ \xi = \frac{Q_A}{|W_S|}  $$

  \section{Legge di Avogadro}
  Volumi uguali, di gas diversi, alla stessa pressione e temperatura, contengono lo stesso numero di molecole $N$
  $$ N = \frac{1}{K_B} \frac{pV}{T} $$
  Il numero di \textbf{moli} $n$ è definito come
  $$ n = \frac{N}{N_A} $$
  Cioè si prende il numero di molecole $N$, e lo si divide per il numero di avogadro $N_A$,
  $$ n = 1 \quad \Rightarrow \quad N = N_A = 6,022 \cdot 10^{23}$$
  La legge di Avogadro espressa con le moli risulta
  $$ n = \frac{1}{R} \frac{pV}{T} $$

  Dove $K_B$ è la costante di $Boltzman$,
  $$ K_B = 1,38 \cdot 10^{-23} \frac{J}{K}$$
  $$ R = K_B N_A = 8,314 \frac{J}{K \cdot mol} $$

  \section{Secondo principio della Termodinamica}
  Il calore non fluisce \textbf{mai} spontaneamente da un corpo ad uno più caldo.
  Mette dei limiti alle possibili trasformazioni di calore in lavoro
  \subsection{Enunciato kelvin-Planck}
  è \textbf{impossibile} realizzare un processo il cui unico risultato sia la trasformazione di calore in lavoro
  \subsection{Enunciato di Clausius}
  è \textbf{impossibile} realizzare un processo il cui unico risultato sia il passaggio di calore da un corpo a uno a temperatura maggiore

  \section{Teorema di Carnot}
  Due macchine termiche, che lavorano a contatto con due sorgenti, $T_1$ e $T_2 > T_1$, una macchia \textbf{generica} $x$ estrae il calore $Q_2$ dalla sorgente $T_2$, restituisce il calore $Q_1$ a $T_1$, e produce del lavoro $W$. \\
  L'altra macchina $r$ è reversibile, ed estrae calore $Q_2'$ da $T_2$, restituisce calore $Q_1'$ a $T_1$ , il lavoro che essa compie è $W$\\
  Allora,
  $$ \eta_{x(T_1, T_2)}  \leq \eta_{r(T_1, t_2)} $$
  Cioè il rendimento di una macchina termica qualsiasi che lavori tra la temperatura $T_1$ e $T_2$ è $\leq$ del rendimento di una macchina reversibile qualsiasi che lavori tra le medesime temperature.
  \subsubsection{Altra formazione}
  $$ \frac{Q_1}{T_1} + \frac{Q_2}{T_2} = 0 $$

  \subsubsection{Corollario}
  Tutte le macchine reversibili, che lavorano tra le stesse temperature hanno lo stesso rendimento
  $$ \eta_{R_1(T_1, T_2)} = \eta_{R_2(T_1, T_2)} $$
  \subsubsection{Osservazioni}
  Il rendimento massimo coincide con quello della macchina reversibile
  $$ \eta_{MAX(T_1,T_2)} = \eta_{R(T_1,T_2)} = 1-\frac{T_1}{T_2} $$

  \section{Teorema di Clausius}
  Estende il Teorema di Carnot alle trasformazioni generiche
  $$ \sum_{j=1}^N \frac{Q_j}{T_j} \leq 0 $$
  Tutti gli scambi di calore con le $N$ sorgenti con cui il sistema è in contatto sommati, devono essere $\leq 0$. è $=0$ se il ciclo (la macchina) è reversibile

  $$ \oint \frac{dQ}{T} \leq 0 $$

  \section{Entropia}
  Data una trasformazione ciclica da $A$ a $B$ nel tratto 1 e da $B$ a $A$ nel tratto 2, se il ciclo è reversibile si ha
  $$ \oint \frac{dQ}{T} = \int_A^B (\frac{dQ}{T})_1 + \int_B^A (\frac{dQ}{T})_2 = 0$$
  ma, visto che il ciclo è reversibile vale anche
  $$ \int_A^B (\frac{dQ}{T})_1 - \int_A^B (\frac{dQ}{T})_2 = 0$$
  Quindi non dipende dal percorso scelto. Abbiamo quindi questa funzione, che si chiama $entropia$
  $$ \int_A^B (\frac{dQ}{T})_{rev} = S(B) -S(A) $$
  La variazione di entropia è una funzione di stato:
  $$ \Delta S_{A \rightarrow B} = \int_A^B (\frac{dQ}{T})_{rev. qualsiasi} $$
  \subsubsection{nota 1}
  Dati un sistema 1 e un sistema 2, il sistema 3 è $= sys1 \cup sys2$, l'entropia
  $$ S_3 = S_1 + S_2 $$
  \subsubsection{nota 2}
  L'entropia è una grandezza estensiva
  \subsubsection{nota 3}
  Per il calcolo, occorre scegliere la reversibile più conveniente
  \subsection{Variazioni di Entropia per Tr. notevoli di Gas Ideali}
  \subsubsection{Isoterma}
  $$ v_i \rightarrow v_f$$
  $$ dQ = \frac{nRT}{V} dV \Rightarrow dS = \frac{dQ}{T} = nR\frac{dV}{V} $$
  $$ \Delta S = nR ln(\frac{V_f}{V_i}) $$

  \subsubsection{Isocora}
  $$ dS = nC_v\frac{dT}{T} $$
  $$ \Delta S = \int_i^f dS = nC_v ln(\frac{T_f}{T_i}) $$

  \subsubsection{Isobara}
  $$ dS = nc_p\frac{dT}{T} $$
  $$ \Delta S = \int_i^f dS = n c_p ln(\frac{T_f}{T_i})$$

  \subsubsection{Adiabatica}
  $$ \Delta S = 0 $$

  \subsubsection{Cambi di fase}
  $$ dS = \frac{\lambda}{T} dm$$
  $$ \Delta S = \frac{\lambda m}{T} $$

  \subsection{Diagrammi T-S}
  $T$ e $S$ sono variabili termo dinamiche, quanto lo sono $p$ e $v$
  \subsection{Teorema dell'entropia}
  Se il sistema è isolato
  $$ dQ = 0 \quad \Rightarrow \quad \int_A^B(\frac{dQ}{T})_X = 0$$
  Segue che
  $$ S_B \geq S_A $$
  $$ \Delta S \geq 0 $$
  e l'$=$ vale solo se la trasf. è reversibile

  \chapter{Elettricità}
  \section{Const}
  \begin{itemize}
    \item Costante magnetica $K_m = 10^{-7} T\cdot m = \frac{\mu_0}{4\pi}$
    \item Costante $K = 8,99 \cdot 10^9 \quad \frac{N\cdot m^2}{C^2} = \frac{1}{4\pi \epsilon_0}$
    \item Costante dielettrica del vuoto $\epsilon_0 = 8,85 \cdot 10^{-12} \frac{C^2}{N \cdot m^2} [= \frac{C^2}{J \cdot m}]$
    \item $1nC = 1 \cdot 10^{-9} C $
  \end{itemize}
  Conduttori: materiali che non si elettrizzano \\
  Isolanti: si elettrizzano, gli isolanti che possono essere:
  \begin{itemize}
    \item Vetri e affini (carica $+$)
    \item Plastiche (carica $-$ )
  \end{itemize}

  Dove $\epsilon_0$ è la \textbf{costante dielettrica del vuoto}
  \subsubsection{La carica elementare}
  è $e = 1,602 \cdot 10^{-19} C$
  \begin{itemize}
    \item la carica $q$ dell'\textbf{elettrone} è $-e$, la sua massa $m_{el} = 0,91 \cdot 10^{-30} kg$
    \item la carica $q$ del \textbf{protone} è $+e$, la sua massa $m_{pr} = 1,67 \cdot 10^{-27} Kg$
    \item la carica $q$ del \textbf{neutrone} è $0$, la sua massa è circa uguale a quella del protone
  \end{itemize}

  \subsection{Campo elettrico in una piana infinita}
  Il campomagnetico $\vec{E}$ di una piana infinita è sempre prerpendicolare alla piana, e vale
  $$ \vec{E} = \frac{\sigma}{2\epsilon_0} $$
  Dove $\sigma$ è la densità di carica

  \section{Forza di attrazione tra due cariche puntiformi}
  Date due cariche $q_1$ e $q_2$ la forza che $q_1$ applica a $q_2$ è pari a
  $$ F_{1 \rightarrow 2} = K \frac{q_1 q_2}{r_{12}^2} $$
  $r_{12}$ è la distanza tra i due oggetti. La forza agisce sempre lungo la congiungente delle due cariche.
  $$u.d.m[q] = [I\cdot T] = 1C (culomb)$$

  \section{Principio di sovrapposizione}
  Date $n$ cariche, che interagiscono su una carica, l'effetto totale sulla carica di prova, è dato dalla sovrapposizione degli effetti delle singole cariche

  \section{Campo elettrico o elettrostatico}
  Il campo elettrico è il rapporto fra la forza che muove la carica e la carica stessa
  $$\vec{E} = \frac{\vec{F}}{q_0} = \frac{K Q}{r^2} \hat{r}$$
  Dove $\vec{E}$ è il campo elettrico, $\vec{F}$ è la forza subita dalla carica, e $q_0$ la carica\\
  Il campo elettrico è \textbf{uscente} se la carica che lo genera $Q$ è $>0$\\
  Il campo elettrico è \textbf{entrante} se la carica che lo genera è $Q<0$
  $$ u.d.m.[E] = [\frac{1N}{C}]$$

  \begin{itemize}
    \item Campo elettrostatico: è generato da altre cariche elettriche ferme
    \item Elettromotore
  \end{itemize}

  \subsection{lavoro di un campo elettrico}
  $$ W_{a \rightarrow b} = \int_a^b \vec{F}\cdot d\vec{s} = \int_a^b q \vec{E} \cdot d\vec{s} = q \int_a^b \vec{E}\cdot d\vec{s} $$

  \subsection{Circuitazione del campo elettrico}
  $$\oint \vec{E} \cdot d\vec{s} = f.e.m.$$
  dove $f.e.m$ è la forza \textbf{elettromotrice}, cioè è uguale alla forza che sta portando la carica dal punto a al punto b (stesso)

  \subsection{caso Elettrostatico}
  $$ W_{AB} = - \Delta U_{AB}$$
  $$ U_{A} = \frac{KQ}{r_A} + const $$
  Dove $U_{AB}$ è l'energia potenziale elettrostatica. Le forze, almeno per la singola carica sono forze conservative e quindi, vale per la carica puntiforme: La circuitazione del campo elettrostatico è $=0$
  $$ \oint \vec{E} \cdot d\vec{s} = 0 $$
  che è equivalente alla conservatività del campo elettrostatico.\\
  Si introduce \textbf{il potenziale} $\Delta V$, definito come la differenza di energia potenziale diviso la carica
  $$ \Delta U \longrightarrow \Delta V = \frac{\Delta U}{q} $$
  Il potenziale elettrostatico si definisce come:
  $$ V_A = \frac{KQ}{r_A} + const^{'}$$
  Dove $const'$ è una costante arbitraria

  \section{Forza Elettrica}
  Se una carica $q$ è in movimento, cè una forza non nulla che la muove, questa forza può essere
  \begin{itemize}
    \item Determinata da altre cariche elettriche, in questo caso si chiama \textbf{forza elettrostatica}
  \end{itemize}

  \section{Linee di forza}
  Dato un campo vettoriale $\vec{v}$, può essere rappresentato graficamente tramite le linee di forza. In ogni punto le linee di forza hanno una tangente, che coincide con la direzione del vettore $\vec{v}$ nel punto $P$
  \begin{itemize}
    \item direzione($\vec{v_P}$) = tangente alle linee di forza nel punto $P$
    \item Verso($\vec{v_P}$) = verso di percorrenza delle linee nel punto $P$
    \item Modulo($\vec{v_P}$) si rappresenta con la densità delle linee
  \end{itemize}

  \section{Flusso di un campo vettoriale}
  Data una grandezza vettoriale $\vec{v}$,  definiamo come flusso della grandezza vettoriale in questione, attraverso una superficie infinitesima $d \Sigma$\\
  Il flusso $d \Phi$ è
  $$ d \Phi = \vec{v}\cdot \hat{n} d \Sigma $$
  dove $\hat{n}$ è il vettore normale alla superficie.\\
  Più in generale per una superficie non infinitesima, si ha la somma di tante superfici infinitesime
  $$ \Phi = \int_\Sigma d\Phi = \int_\Sigma \vec{v} \cdot \hat{n} d\Sigma $$

  \section{Angolo solido}
  Area della calotta sferica, normalizzata al quadrato del raggio della sfera, l'angolo solido è
  $$ \omega = \frac{S}{R^2} $$
  dove $S$ è la superficie "di taglio" sempre ortogonale

  \section{teorema di Gauss}
  Nel caso in cui le cariche siano \textbf{interne:}
  Il flusso del campo elettrico attraverso una superficie chiusa è uguale alla somma delle cariche interne a tale superficie diviso $\epsilon_0$
  $$ \Phi_{\Sigma} = \oint \vec{E} \cdot \hat{n} d\epsilon = \sum_i \frac{q_i}{\epsilon_0} $$
  \paragraph{Nel caso di cariche esterne}
  Si ha che il flusso è nullo
  $$ d\Phi_{d\omega} = 0$$
  Per il flusso tramite la superficie $\omega$


  \section{Conduttori}
  I conduttori si definiscono tali, perchè al loro interno contengono cariche libere. Una carica si definisce libera, quando, sotto l'azione di un campo elettromagnetico esterno essa si muove più o meno liberamente
  \begin{itemize}
    \item Soluzioni elettrolitiche
      \item gas ionizzati
    \item metalli
  \end{itemize}

  \subsection{Conduttore in Equilibrio}
  Un conduttore si dice in equilibrio quando tutte le cariche al suo interno sono in equilibrio\\
  Cioè $\forall$ carica $q_i$
  $$\vec{v_i} = \vec{0} \Rightarrow \vec{F_i} = \vec{0} \Rightarrow \vec{E_i} = \vec{0}$$
  Definizione "media". Il campo elettrico è nullo (perchè le cariche non si muovono)

  \subsubsection{Primo corollario}
  Per un conduttore in equilibrio vale $q_{intern} =0$. Se c'è una carica, si distribuisce sulla superficie del conduttore
  \subsubsection{Secondo corollario}
  Dati due punti $A, B$ all'interno del conduttore, la differenza di potenziale tra $A$ e $B$
  $$V_A - V_B = \int_A^B \vec{E}\cdot d\vec{s}=0 $$
  $\forall A,B$ interni al conduttore, $A=B$. Cioè tutti i punti di un conduttore in equilibrio sono allo stesso potenziale
  \subsubsection{terzo corollario}
  \( \vec{E} \neq \vec{0} \) solo appena fuori dalla superficie
  $$ \vec{E}(\vec{r}) = \frac{\sigma(\vec{r})}{\epsilon_0} \hat{n} $$
  QUindi, il campo elettrico appena fuori dalla superficie di un conduttore, in una posizione $r$ è uguale alla densità di carica nella data posizione $r$ diviso $\epsilon_0$, e diretto normalmente alla superficie

  \subsection{Schermo elettrostatico}
  Immaginando un conduttore cavo\\
  Una perturbazione esterna che redistribuisce $q_{EF}$ non sarà percepibile all'interno del conduttore e quindi su tutti i conduttori interni.\\
  Ogni perturbazione interna non sarà misurabile all'esterno

  \section{Capacità del conduttore}
  è il rapporto tra la carica $q$ depositata tra la superficie del conduttore, e il potenziale $V$ in cui il conduttore va, per effetto di questa carica. descrive la capacità di un conduttore di ospitare le cariche e di portarsi ad un certo potenziale
  $$ \mathbb{C} = \frac{q}{V} $$
  $$u.d.m.[\mathbb{C}] = \frac{1C}{V} = 1F$$
  dove $ F $ è il Farad

  \section{Condensatori}
  Quando due conduttori sono ininduzione completa, si chiamano condensatore ($R_1$ e $R_2$ sono i raggi del condensatore sferico)

  $$ \Delta V = Kq (\frac{1}{R_1} - \frac{1}{R_2}) $$

  \subsection{Capacità di un condensatore}
  $$ C= \frac{q}{\Delta V} = 4 \pi \epsilon_0 \frac{R_1 R_2}{R_2 - R_1} $$

  Se le superfici delle armature sono abbastanza vicine, la capacità è
  $$C = \epsilon_0 \frac{S}{h}$$
  Dove $S$ è la superficie, $h$ è pari a $R_2 = R_1 + h$ con $h<<R_1$

  \subsection{Energia elettrostatica nel condensatore}
  $$ \Delta U = W_{TOT} = \frac{Q^2}{2C} = \frac{1}{2} C \Delta V^2 = \frac{1}{2} Q \Delta V $$

  \subsection{Campo elettrico nel condensatore}
  $$ |\vec{E}| = \frac{\sigma}{\epsilon_0} $$
  dove $\sigma$ è la densità superficiale di carica

  \subsection{Induzione Completa}
  Quando, tutte le linee di campo elettrico nascono da un conduttore e muoiono nell'altro conduttore. Cioè non esiste una linea di campo elettrico che parte da un conduttore che non vada a finire nell'altro.

  \subsection{Condensatore piano}
  Due armature piane, parallele, con carica $+q$ e $-q$.
  Il campo elettrico è pari a \( E = \frac{\sigma}{2\epsilon_0} \)
  La somma dei vettori dei due campi elettrici è pari a $0$ all'esterno delle due armature, mentre all'interno è pari a $E$
  $$ C = \epsilon_0 \frac{S}{h} $$

  \subsubsection{Effetti di bordo}
  Al limite delle armature ci sono delle linee che saranno curvate, trascurate per la debole intensità.

  \subsection{Sistemi di condensatori}
  Dati due condensatori, $C_1$ e $C_2$ collegati a una differenza di potenziale $\Delta V$,
  $$ \Delta V = \Delta V_1 + \Delta V_2 $$
  $$ C_1 = \frac{q_1}{\Delta V_1} $$
  $$ C_2 = \frac{q_2}{\Delta V_2} $$

  \subsubsection{In serie}
  Il condensatore "risultante" sarà
  $$ \frac{1}{C} = \frac{1}{C_1} + \frac{1}{C_2} $$

  \subsubsection{In parallelo}
  $$ C = C_1 + C_2 $$

  \subsection{Energia nel condensatore}
  L'energia elettrostatica all'interno del condensatore è
  $$ \Delta U_{el} = \frac{1}{2} \epsilon_0 E^2 \tau $$
  Dove $\tau$ è il volume tra le due armature\\
  La densità volumica di energia è
  $$ u = \frac{\Delta U_{el}}{h} = \frac{1}{2} \epsilon_0 E^2 $$

  \subsection{Forza tra le armature}
  La forza tra le due armature del condensatore è definita come
  $$ F = -\frac{dU}{dS} $$
  $$ F = -\frac{d}{dS}[\frac{1}{2} \epsilon_0 E^2 \tau] $$

  \subsection{Pressione elettrostatica}
  Indipendente dal fatto che il condensatore sia isolato o connesso a un generatore, non dipende dalla geometria del condensatore è un equazione valida in "generale", \textbf{coincide con la definizione di densità elettrostatica associata al campo elettrico}
  $$ \frac{1}{2} \epsilon_0 E^2 (=u) $$

  \subsubsection{Condensatore isolato}
  La pressione elettrostatica tra due armature di un condensatore coincide con la densità volumica di energia
  $$ P_{el} = \frac{F}{S} = -\frac{1}{2} \epsilon_0 E^2 $$

  \subsubsection{Condensatore e generatore}
  $$ d(\Delta U_{el}) = -\frac{1}{2} \epsilon_0 (\frac{\Delta V}{h})^2 S dh $$
  $$ d(\Delta U_{gen}) = \epsilon_0 (\frac{\Delta V}{h})^2 S dh $$
  e quindi, la variazione totale sarà
  $$ d(\Delta U) =\frac{1}{2} \epsilon_0 (\frac{\Delta V}{h})^2 S dh $$
  $$ F = \frac{-d(\Delta U)}{dh} = -\frac{1}{2} \epsilon_0 (\frac{\Delta V}{h})^2 S $$
  $$ P = \frac{F}{S} = -\frac{1}{2} \epsilon_0 E^2 $$

  \section{Conduzione elettrica}
  Per esempio, il rame $Cu$ ha 1 elettrone "libero" per atomo. Calcoliamo quanti elettroni sono liberi per unità di volume
  $$ n = \frac{N_A \rho}{ A} = 8,49 \cdot 10^{28} \frac{e^-}{m^3} $$
  Elettroni a metro cubo. la velocità media degli elettroni (per un volume abbastanza grande) è
  $$\vec{v_m} = \vec{0}$$
  Se il conduttore è sottoposto ad un campo elettrico esterno, la velocità media degli elettroni passa a quella di "drift":
  $$ \vec{v_m} = \vec{0} \longrightarrow \vec{v_m} = \vec{v_d} $$
  Vettore densità di corrente elettrica, pari alla densità di cariche libere, per velocità di drift (o deriva), - la stessa cosa per le cariche negative
  $$ \vec{J_e} = n_+ e \vec{v_{d+}} - n_- e \vec{v_{d-}}$$

  \subsection{Corrente elettrica}
  Si definisce come l'integrale verso la sezione del vostro filo
  $$ i = \int_S n e \vec{v} \cdot \hat{n} dS = \frac{dq}{qt}$$
  $$ u.d.m[i]=\frac{1C}{1s}= 1A $$

  \subsection{Principio di conservazione della carica elettrica}
  Data una superficie chiusa, la carica totale che passa attraverso questa superficie, è uguale alla variazione delle cariche nella superficie nel tempo. (tanta corrente entri nel volume tanta ne esce)
  $$ i_{tot} = \oint \vec{J} \cdot \hat{n} dS = -\frac{dq_{int}}{dt} $$

  \subsection{Legge di OHM}
  La velocità di deriva, è proporzionale al campo elettrico
  $$ \vec{v_d} = \alpha \vec{E} $$
  Anche scritta
  $$ \vec{J} = \frac{1}{\rho} \vec{E} $$
  La costante $\rho$ si chiama resistività, è una costante intensiva, e dipende dal materiale che stiamo utilizzando.
  La resistività cresce in proporzione alla temperatura
  $$ \rho(t) = \rho{20}(1 + \alpha \Delta t) $$
  Dove $\rho{20}$ è la resistività misurata a 20 gradi celsius\\
  Cioè la legge di OHM lega il campo elettrico alla corrente tramite la costante di resistività $\rho$
  $$ \vec{E} = \rho \cdot \vec{J} $$


  $$ i = \frac{ES}{\rho} = Eh \frac{S}{\rho h} = \Delta V \frac{1}{R} $$
  $R$ Si chiama resistenza elettrica,
  $$ R = \rho \frac{h}{S} $$
  Dove $h$ è la lunghezza, e $S$ la sezione

  La legge di OHM per i conduttori metallici sarà quindi
  $$ i = \frac{\Delta V}{R} $$
  $$ u.d.m[R] = \frac{1V}{1A} = 1 \Omega $$

  \subsection{Effetto Joule}
  La potenza, $P$
  $$ P=\frac{dW}{dt} = \frac{dq\Delta V}{dt} = \Delta V \cdot i$$

  \paragraph{Per circuiti dove vale la legge di OHM}
  Si ha
  $$ P = R I^2 $$


  \subsection{Resistenze in serie}
  Consideriamo $R_1$ e $R_2$ in serie, la corrente $i$ che le attraversa è la stessa, quindi
  $$ i R_1 = \Delta V_1 $$
  $$ i R_2 = \Delta V_2 $$
  Dove $\Delta V_1$ è la differenza di potenziale ai capi di $R_1$, stessa cosa per $\Delta V_2$\\
  Quindi,
  $$ i R^* = \Delta V \Rightarrow R^* = R_1 +R_2 $$

  \subsection{Resistenze in parallelo}
  Consideriamo $R_1$ e $R_2$ in parallelo, la corrente $i$ si divide nei due capi in $i_1$ e $i_2$. In questo caso, $\Delta V = \Delta V_1 = \Delta V_2$ quindi
  $$ i_1 R_1 = \Delta V $$
  $$ i_2 R_2 = \Delta V $$
  Quindi,
  $$ \frac{1}{R*} = \frac{1}{R_1} + \frac{1}{R_2} $$

  \section{Forza Elettro Motrice}
  $$ f.e.m. = \oint \vec{E} \cdot d\vec{s} = R_T i \Rightarrow i \neq 0 \iff f.e.m. \neq 0$$
  $R_T$ resistenza totale circuito\\
  Deve quindi esistere una resistenza interna al generatore tale che
  $$ f.e.m. = R_T i = (R_{load} + r)i $$
  $$ V_A - V_B = f.e.m. -ri $$
  Dove $R_{load}$ è la resistenza del "circuito" senza il generatore, e $r$ è la resistenza del generatore

  \section{Interazione elettromagnetica}
  Magnetite $F_eO \cdot Fe_2O_3$. I \textbf{poli magnetici} sono le estremità di un oggetto (in magnetite)\\
  Esistono due tipi di "carica" magnetica: positive e negative.\\
  Ciascun magnete contiene sempre due poli opposti\\
  La magnetizzazione: si possono produrre magneti artificiali (calamite) tramite contatto coi magneti naturali.\\
  Il geomagnetismo: lasciato libero di muoversi, si orienta lungo direzioni secondo il meridiano terrestre locale. il polo che punta verso il nord è chiamato polo nord con carica positiva. il polo sud viceversa.\\
  $$F = K_m \frac{q_{m1}q_{m2}}{r^2}$$
  Magnete spezzato: se spezziamo un magnete, non ottengo due oggetti con cariche opposte, ma ottengo due nuovi magneti. Questo fenomeno è valido per qualsiasi scala. Non esiste il monopolo magnetico\\
  Le correnti generano campi elettromagnetici
  \subsection{Induzione magnetica}
  $\vec{B}$ è l'induzione magnetica $udm[B] = 1T$ (un Tesla)
  \subsection{Campo magnetico}
  $$ \oint_S \vec{B}\cdot \hat{n} d\Sigma = 0 $$
  Campo magnetico solenoidale. (solenoidale quando il flusso del campo attraverso una qualsiasi superficie chiusa è uguale a 0)

  \section{Forza di Lorentz}
  è la forza che sperimenta una carica $q$, in moto con velocità $\vec{v}$, quando essa entra in una regione con campo magnetico $\vec{B}$
  $$ \vec{F_L} = q\vec{v} \times \vec{B} $$
  $$ |\vec{F}| = F_L = qvB sin\theta_{vB} $$
  $$ [B] = [\frac{F}{Q\frac{L}{T}}] = [\frac{F\cdot L}{I\cdot L^2}] = [\frac{E}{I\cdot L^2}] $$

  \section{Forze che agiscono su una carica}
  $$ \vec{F} = q(\vec{E} + \vec{v} \times \vec{B}) $$

  \section{Forza su un tratto di filo percorso da cariche}
  $$ dF = -e dN \vec{v_d} \times \vec{B}$$
  dove $dN$ è il numero di elettroni. $dN = nd\tau$ che è la densità volumica\\
  La riedizione della forza di Lorentz, invece che per una carica puntiforme per un tratto di filo infinitesimo
  $$ d\vec{F} = i d\vec{S} \times \vec{B} $$+++
  che è anche chiamata seconda legge elementare di Laplace

  \section{Principio di equivalenza di Ampere}
  Principio che pone l'equivalenza tra spire attraversate da corrente e magneti:
  $$ \vec{m} = i \Sigma \hat{n} $$
  è il momento magnetico della spira. Permette di pensare alla spira come ad un bipolo magnetico.\\
  Versione infinitesimale, che è anche il principio di equivalenza di ampere:
  $$ d\vec{m} = id\Sigma \hat{n} $$

  \section{Prima legge elementare di Laplace}
  Le correnti possono generare campi magnetici.
  Si ha $id\vec{s}$ un tratto lungo un filo, mi interessa sapere qual'è il campo magnetico esercitato dal piccolo segmento $id\vec{s}$ nel punto $P$ generico. Si ha anche la congiungente $\hat{u}_r=\hat{r}$ dal punto di sorgente del campo magnetico al punto $P$.
  $$ d\vec{B} = K_m \frac{id\vec{s} \times \hat{u}_r}{r^2} $$
  \subparagraph{Nota}
  La prima legge elementare di Laplace è uno strumento matematico, non ha senso fisico: è impossibile misurare il contributo di $id\vec{s}$ senza misurare il contributo di tutto il "circuito" in quanto la corrente non si può spezzare in più parti.
  \paragraph{Forma generale con conduttore filiforme}
  $$ \vec{B}(P) = \oint_C d\vec{B} = \oint_C \frac{\mu_0}{4\pi} \frac{id\vec{s} \times \hat{u_r}}{r^2}$$
  Dove $C$ è il filo
  \paragraph{Forma generale con conduttore non filiforme}
  $$ \vec{B}(P) = \frac{\mu_0}{4\pi} \int_V \vec{J}\frac{\hat{u_\tau} \times \hat{u_r}}{r^2} d\tau $$

  \section{Legge di Biot-Savart}
  Il campo magnetico $\vec{B}$ generato a distanza $R$, da un filo indefinito percorso dalla corrente $i$ è
  $$ \vec{B}(R) = \frac{\mu_0 i}{2 \pi R} \hat{\theta} $$
  Dove $\hat{\theta}$ è il versore del campo magnetico

  \section{Legge di Ampere}
  $$ \tau_C (\vec{B}) = \oint_C \vec{B} \cdot d\vec{l} = \mu_0 i_C $$
  Dove $i_C$ sono le correnti concatenate (interne) al circuito

  \subsection{Equazioni di Maxwell}
  Per campi elettrici e magnetici statici (che non variano nel tempo)
  $$ \oint_S \vec{E}\cdot \hat{n} dS = \frac{q_{int}}{\epsilon_0} $$
  $$ \oint_S \vec{B}\cdot \hat{n} dS = 0 $$
  $$ \oint_C \vec{E}\cdot d\vec{l} = 0 $$
  $$ \oint_C \vec{B} \cdot d\vec{l} = \mu_0 i_C $$

  \section{Legge di Faraday-Neumann-Lenz}
  Ogni variazione del flusso nel tempo genera una f.e.m.
  $$f.e.m = \quad -\frac{d\Phi(\vec{B})}{dt} $$
  $$ i = -\frac{f.e.m.}{R} $$

  L62
L 143
\end{document}
