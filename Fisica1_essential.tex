\documentclass[a4paper]{report}

\usepackage[top=25mm,bottom=25mm]{geometry}
\usepackage[utf8]{inputenc}
\usepackage[italian]{babel}
\usepackage[T1]{fontenc}

\usepackage{amssymb}
\usepackage{mathtools}
\usepackage{graphicx}

\usepackage{hyperref}

\title{Dispense essenziali di Fisica 1}
\author{Matteo Bitussi \\ Laurea in Informatica, Unitn}
\date{Anno accademico 2019-2020}

\begin{document}
  \maketitle

  \tableofcontents

  \section*{Introduzione}
  Questa dispensa è pensata per raccogliere le informazioni essenziali necessarie per lo svolgimento degli esercizi durante l'anno e/o per l'esame finale. Per questo motivo non saranno approfondite e non potranno sostituire quelle fornite dal professore.

  \chapter{Cinematica del punto}
  \section{Moto rettilineo}
  \subsection{Velocità media}
  La velocità media $v_m$ del punto è il rapporto tra spostamento e tempo:
  \[ v_m = \frac{\Delta x}{\Delta t} = \frac{x_2 - x_1}{t_2 - t_1} \]
  Essa coincide con la definizione matematica di valor medio di una funzione in un dato intervallo
  \[ v_m = \frac{1}{t-t_0} \int_{t_0}^t v(t) dt \]
  \subsection{Velocità istantanea}
  La velocità istantanea  rappresenta la rapidità di variazione temporale della posizione nell'istante $t$ considerato. è data dalla derivata dello spazio rispetto al tempo
  \[ v = \frac{dx}{dt} \]
  \subsection{Leggi orarie}
  è la relazione generale che permette il calcolo dello spazio percorso nel moto rettilineo, qualunque sia il tipo di moto
  \[ x(t) = x_0 + \int_{t_0}^t v(t) dt \]
  $x_0$ rappresenta la posizione iniziale del punto, occupata nell'istante $t_0$
  \\
    Data l'accelerazione $a(t)$ si può ottenere la velocità $v(t)$ cioè vale la relazione
  \[ v(t) = v_0 + \int_{t_0}^t a(t) dt\]

  \subsection{Moto Rettilineo Uniforme}
  Nel MRU la velocità $v$ è costante
  \[ x(t) = x_0 + v(t-t_0)\]

  \subsection{Accellerazione media}
  Se tra gli istanti di tempo $t_1$ e $t_2$ la velocità varia da $v_1$ a $v_2$, si definisce \textbf{accellerazione media} del punto, il rapporto tra la variazione di velocità e l'intervallo di tempo
  \[ a_m = \frac{v_2-v_1}{t_2-t_1} = \frac{\Delta v}{\Delta t} \]

  \subsection{Accellerazione istantanea}
  \[ a = \frac{dv}{dt} = \frac{d^2x}{dt^2} \]

  \subsection{Moto rettilineo uniformemente accellerato}
  Se l'accellerazione di un moto è costante, questo si dice uniformemente accellerato, e la dipenmdenza della velocità dal tempo è lineare

  \[ v(t) = v_0 + a(t-t_0) \]
  \[ x(t) = x_0 + v_0(t-t_0) + \frac{1}{2} a(t-t_0)^2  \]

  \subsection{Moto armonico semplice}
  Un punto segue un moto armonico semplice quando la legge oraria è definita dalla relazione
  \[ x(t) = A sen(\omega t + \phi) \]
    Dove $A, \omega, \phi$ sono grandezze costanti: $A$ è detta \textbf{ampiezza del moto}, $\omega t + \phi$ \textbf{fase del moto}, $\phi$ \textbf{fase iniziale}, $\omega$ \textbf{pulsazione}.

    Il MAS è quindi un moto vario, dove tutte le grandezze cinematiche che lo descrivono $(x(t), v(t), a(t))$ variano nel tempo.
    Il periodo $T$ è
    \[ T = \frac{2\pi}{\omega} \]

    Si definisce \textbf{frequenza} $\nu$ del moto, il numero di oscillazioni in un secondo
    \[ \nu = \frac{1}{T}=\frac{\omega}{2\pi} \]
  \subsubsection{Velocità nel MAS}
  La velocità del punto che si muove con moto armonico si ottiene derivando $x(t)$:
  \[ v(t)=\frac{dx}{dt} = \omega A cos(\omega t + \phi) \]




\end{document}
